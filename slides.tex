\documentclass[aspectratio=169,xcolor=dvipsnames]{beamer}
\usetheme{SimplePlus}

\usepackage{stmaryrd}
\usepackage{hyperref}
\usepackage{graphicx}
\usepackage{booktabs}

\title[short title]{Anisotropic diffusion for denoising and edge detection}

\author[] {Benjamin Loison, Paul Robert}

\date{January 18, 2022}

\begin{document}

\begin{frame}
    \titlepage
\end{frame}

\begin{frame}{Goal}
\begin{figure}
    \centering
    \includegraphics[scale=0.15]  {brainArrow.png} % used to be brain
\end{figure}
    Removing noise by blurring only the irrelevant parts of the image
\end{frame}

\begin{frame}{Isotropic diffusion}

    \begin{itemize}
        \pause
        \item Gaussian kernels is a way to denoise images 
        \pause
        \item $K_\sigma * I $
        \pause 
        \item solution to an heat equation
        \pause
        \item 
    $\partial_t I = \Delta I \quad
    I(x,0)=f(x)$
        \pause
        \item $I(x,t)=K_{\sqrt{2t}} * f(x)$  $(x>0)$  
        \pause
        \item Gaussian smooting equivalent to a diffusion process until $T= \sigma^2 / 2 $
        \pause
        \item independant of directions: edges are ignored 
    \end{itemize}
\end{frame}

\begin{frame}{Anisotropic diffusion}
    %\vspace*{-1.5cm}
    \begin{itemize}
        \pause
        \item Causality criterion: any feature at a coarse level is the result of one at a finer level
        \pause
        \item Extremum principle: extremum only belong to the original image
        \pause
        \item Piecewise smoothing: smooth within regions, not across
        \hspace*{-1cm}\includegraphics[width=15cm]{cleanTaj} % vertical space would be appreciated
    \end{itemize}
\end{frame}

\begin{frame}{An iterative process}
    \centering
    \includegraphics[width=7cm]{7}
\end{frame}

\begin{frame}{Relation to physics}
    \begin{itemize}
        \pause
        \item Fick's law: $j = -D\cdot\nabla u$
        \pause
        \item Continuity equation: $\partial_t u = -\texttt{div } j$
        \pause
        \item $j$ expression changed in the continuity equation gives the heat equation: $\partial_t u = \texttt{div} (D\cdot\nabla u)$
    \end{itemize}
\end{frame}

\begin{frame}{Perona and Malik's flux functions}
    \begin{itemize}
        \item Heat equation rewritten: $\partial_t I = \texttt{div} (c(x,y,t)\nabla I)$
        \pause
        \item Anisotropic diffusion equation: $\partial_t I = c(x, y, t)\Delta I+\nabla c\cdot\nabla I$\\(the case $c = 1$ is the isotropic case)
        \pause
        \item Flux functions:
        \begin{itemize}
            \item quadratic: $c(||\nabla I||) = \frac{1}{1+\left(\frac{||\nabla I||}{K}\right)^2}$
            \item exponent: $c(||\nabla I||) = e^{-\left(\frac{||\nabla I||}{K}\right)^2}$
        \end{itemize}
    \end{itemize}
\end{frame}

\begin{frame}{Numerical methods}
    Reminder of the anisotropic diffusion equation: $\partial_t I = c(x, y, t)\Delta I+\nabla c\cdot\nabla I$
    \pause
    \centering
    $I_{i,j}^{t+1}=I_{i,j}^t + \lambda(c_N \cdot \triangledown_N I + c_S \cdot \triangledown_S I + c_E \cdot \triangledown_E I + c_W \cdot \triangledown_W I)_{i,j}^t$
    
    \pause
    $$
      \begin{split}
        \triangledown_N I_{i,j} &= I_{i-1,j}-I_{i,j}\\
        \triangledown_S I_{i,j} &= I_{i+1,j}-I_{i,j}\\
        \triangledown_E I_{i,j} &= I_{i,j+1}-I_{i,j}\\
        \triangledown_W I_{i,j} &= I_{i,j-1}-I_{i,j}
      \end{split}
    \quad\quad
      \begin{split}
        c_{N_{i,j}}^t &= g(||(\nabla I)_{i+\frac{1}{2},j}^t||)\\
        c_{S_{i,j}}^t &= g(||(\nabla I)_{i-\frac{1}{2},j}^t||)\\
        c_{E_{i,j}}^t &= g(||(\nabla I)_{i,j+\frac{1}{2}}^t||)\\
        c_{W_{i,j}}^t &= g(||(\nabla I)_{i,j-\frac{1}{2}}^t||)
      \end{split}$$
      
       $g$ the function such that $c(x,y,t)=g(I(x,y,t))$
\end{frame}


\begin{frame}{Comparison with Canny detector}
    \centering
    \includegraphics[width=14cm]{canny}
\end{frame}

\begin{frame}{Relation with Bilateral filter}
\begin{itemize}
    \item $BF[I](x)=\frac{1}{W_p}\sum_{x_i\in\Omega} I(x_i) G_R (I(x_i) - I(x)) G_S(x_i - x) $
    \pause
    \item $I^{t+1}(x) = \frac{\sum_{x_i\in \Omega}I(x_i)w^t(x,x_i)}{\sum_{x_i\in\Omega}w^t (x,x_i)}$
    \pause
    \item $w^t(x_i,x)=G_R(I(x_i) - I(x))G_S(x_i - x)$
    \pause
    \item $I^{t+1}(x)=\sum_{x_i\in\Omega}c(x,x_i,t)I(x_i)$ with $\sum c(x,x_i,t)=1$
    \pause
\end{itemize}
    
\end{frame}

\begin{frame}{Relation with Bilateral filter}
\begin{itemize}
    \item 1-D case with $\Omega = \llbracket x-1;x+1\rrbracket$
    \pause
    \item $I^{t+1} - I^t(x) = c(x+1,t)(I^t(x+1) - I^t(x)) + c(x-1,t)(I^t(x-1) - I^t(x))$
    \pause
    \item Anisotropic diffusion equation: $\partial_t I =  \texttt{div} (c(x,t)\nabla I)$
    \pause
    \item $c(x_1,x_2)= \frac{w^t(x_1,x_2)}{W_p}=\lambda e^{-\left(\frac{||\nabla I||}{K}\right)^2}$ with $K = \sqrt{2}\sigma_R$
    \pause
    \item Anisotropic diffusion can be viewed as a particular case of bilateral filtering
\end{itemize}
    
\end{frame}

\begin{frame}{Conclusion}
    \begin{itemize}
        \item Anisotropic diffusion offers an edge sensitive noise reduction method 
        \pause
        \item Easy to understand, implement, and compute
        \pause
        \item Provides good results, but other methods exist 
    \end{itemize}
\end{frame}

\begin{frame}{References}
    \footnotesize{
        \begin{thebibliography}{99}
            \bibitem[Smith, 2012]{p1} Pietro Perona and Jitendra Malik
            \newblock Scale-Space and Edge Detection Using Anisotropic Diffusion
            \newblock \emph{http://image.diku.dk/imagecanon/material/PeronaMalik1990.pdf}
        \end{thebibliography}
        \begin{thebibliography}{99}
            \bibitem[Smith, 2012]{p1} Joachim Weickert
            \newblock Anisotropic Diffusion in Image Processing
            \newblock \emph{https://www.mia.uni-saarland.de/weickert/Papers/book.pdf}
        \end{thebibliography}
        \begin{thebibliography}{99}
            \bibitem[Smith, 2012]{p1} Manasi Datar
            \newblock Project 2: Anisotropic Diffusion
            \newblock \emph{http://www.cs.utah.edu/\~manasi/coursework/cs7960/p2/project2.html}
        \end{thebibliography}
        \begin{thebibliography}{99}
            \bibitem[Smith, 2012]{p1} Danny Barash
            \newblock Bilateral Filtering and Anisotropic Diffusion: Towards a Unified Viewpoint
            \newblock \emph{https://www.hpl.hp.com/techreports/2000/HPL-2000-18R1.pdf}
        \end{thebibliography}
    }
\end{frame}

\begin{frame}
    \Huge{\centerline{\textbf{Any questions ?}}}
\end{frame}

\end{document}
